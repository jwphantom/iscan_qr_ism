\documentclass[a4paper,12pt]{article}
\usepackage[french]{babel}
\usepackage{hyperref}
\usepackage{listings}
\usepackage{xcolor}

\definecolor{codegray}{rgb}{0.95,0.95,0.95}
\definecolor{darkgreen}{rgb}{0.0,0.5,0.0}

\lstdefinestyle{mystyle}{
    backgroundcolor=\color{codegray},   
    commentstyle=\color{darkgreen},
    keywordstyle=\color{blue},
    numberstyle=\tiny\color{gray},
    stringstyle=\color{red},
    basicstyle=\ttfamily\footnotesize,
    breakatwhitespace=false,         
    breaklines=true,                 
    captionpos=b,                    
    keepspaces=true,                 
    numbers=left,                    
    numbersep=5pt,                  
    showspaces=false,                
    showstringspaces=false,
    showtabs=false,                  
    tabsize=2
}

\lstset{style=mystyle}

\title{Documentation de l'API de Vérification QR Code et Photo}
\author{Équipe de Développement}
\date{\today}

\begin{document}

\maketitle

\section{Introduction}
Cette documentation explique comment le serveur backend en \textbf{Python (Flask)} doit gérer les requêtes de vérification envoyées par une application \textbf{Flutter}. L'application envoie des \textbf{QR codes} et/ou des \textbf{photos} pour authentification.

\section{Endpoint attendu}
L'application Flutter envoie une requête HTTP \textbf{POST} vers :
\begin{verbatim}
POST http://<IP>:<PORT>/validate_access
\end{verbatim}
L'adresse IP et le port sont configurables dans l'application.

\section{Format de la requête}

L'application supporte trois modes de vérification :
\begin{itemize}
    \item \textbf{QR Code seulement} : \texttt{scan\_mode="qr"}
    \item \textbf{Photo seulement} : \texttt{scan\_mode="photo"}
    \item \textbf{Les deux (par défaut)} : \texttt{scan\_mode="both"}
\end{itemize}

Les données envoyées peuvent inclure :
\begin{itemize}
    \item \texttt{scan\_mode} : \texttt{"qr"}, \texttt{"photo"} ou \texttt{"both"}.
    \item \texttt{encrypted\_data} : chaîne encodée (nécessaire pour QR Code).
    \item \texttt{photo} : fichier image (nécessaire pour la vérification photo).
\end{itemize}

\section{Exemple de requête HTTP}

Sans photo :
\begin{lstlisting}
POST /validate_access HTTP/1.1
Host: <IP>:<PORT>
Content-Type: multipart/form-data; boundary=----Boundary

------Boundary
Content-Disposition: form-data; name="scan_mode"

qr
------Boundary
Content-Disposition: form-data; name="encrypted_data"

EXEMPLE_QR_CODE
------Boundary--
\end{lstlisting}

Avec photo :
\begin{lstlisting}
POST /validate_access HTTP/1.1
Host: <IP>:<PORT>
Content-Type: multipart/form-data; boundary=----Boundary

------Boundary
Content-Disposition: form-data; name="scan_mode"

photo
------Boundary
Content-Disposition: form-data; name="photo"; filename="photo.jpg"
Content-Type: image/jpeg

[BLOB DE L'IMAGE]
------Boundary--
\end{lstlisting}

\section{Traitement côté Backend (Flask)}

\textbf{Installation des dépendances} :
\begin{lstlisting}[language=bash]
pip install flask flask-cors opencv-python numpy
\end{lstlisting}

\textbf{Code du serveur Flask} :
\begin{lstlisting}[language=python]
from flask import Flask, request, jsonify
import cv2
import numpy as np

app = Flask(__name__)

@app.route('/validate_access', methods=['POST'])
def validate_access():
    try:
        scan_mode = request.form.get('scan_mode', 'both')

        if scan_mode in ['qr', 'both']:
            qr_data = request.form.get('encrypted_data')
            if not qr_data:
                return jsonify({"error": "QR Code manquant"}), 400
            if qr_data != "EXEMPLE_QR_CODE_VALID":
                return jsonify({"error": "QR Code invalide"}), 403

        if scan_mode in ['photo', 'both']:
            if 'photo' not in request.files:
                return jsonify({"error": "Photo manquante"}), 400
            file = request.files['photo']
            np_image = np.frombuffer(file.read(), np.uint8)
            img = cv2.imdecode(np_image, cv2.IMREAD_COLOR)

            if not is_valid_face(img):
                return jsonify({"error": "Photo non reconnue"}), 403

        return jsonify({"status": "Access granted"}), 200

    except Exception as e:
        return jsonify({"error": str(e)}), 500

def is_valid_face(image):
    face_cascade = cv2.CascadeClassifier(cv2.data.haarcascades + 'haarcascade_frontalface_default.xml')
    gray = cv2.cvtColor(image, cv2.COLOR_BGR2GRAY)
    faces = face_cascade.detectMultiScale(gray, scaleFactor=1.1, minNeighbors=5)
    return len(faces) > 0

if __name__ == '__main__':
    app.run(host="0.0.0.0", port=5000, debug=True)
\end{lstlisting}

\section{Réponses attendues du serveur}

Le backend doit répondre clairement à l'application Flutter :

\begin{center}
\begin{tabular}{|c|l|l|}
\hline
\textbf{Code HTTP} & \textbf{Description} & \textbf{Réponse JSON} \\
\hline
\texttt{200} & Accès autorisé & \texttt{\{"status": "Access granted"\}} \\
\hline
\texttt{400} & Requête invalide & \texttt{\{"error": "QR Code manquant"\}} \\
\hline
\texttt{403} & Accès refusé & \texttt{\{"error": "QR Code invalide"\}} \\
\hline
\texttt{500} & Erreur interne & \texttt{\{"error": "Message d'erreur"\}} \\
\hline
\end{tabular}
\end{center}

\section{Interaction entre Flutter et le Backend}

\begin{enumerate}
    \item L'utilisateur \textbf{scanne un QR Code} et/ou \textbf{prend une photo}.
    \item L'application Flutter envoie une \textbf{requête HTTP POST} au serveur.
    \item Le backend \textbf{vérifie les données} :
    \begin{itemize}
        \item QR Code → comparaison avec une base de données.
        \item Photo → reconnaissance faciale (via OpenCV).
    \end{itemize}
    \item Le backend \textbf{retourne une réponse appropriée} :
    \begin{itemize}
        \item \texttt{200} : Accès autorisé.
        \item \texttt{403} : Accès refusé.
        \item \texttt{400} : Requête incorrecte.
        \item \texttt{500} : Erreur interne.
    \end{itemize}
\end{enumerate}

\section{Améliorations possibles}
\begin{itemize}
    \item Vérification avancée des QR Codes (base de données).
    \item Utilisation de l'IA pour la reconnaissance faciale.
    \item Ajout de logs et monitoring des accès.
\end{itemize}

\section{Conclusion}
Cette documentation décrit comment l'API de vérification d'accès doit être implémentée en Python/Flask pour communiquer avec l'application Flutter. En suivant ces instructions, le serveur pourra gérer efficacement les authentifications basées sur QR Code et reconnaissance faciale.

\end{document}
